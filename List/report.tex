\documentclass[UTF8]{ctexart}
\usepackage{geometry, CJKutf8}
\geometry{margin=1.5cm, vmargin={0pt,1cm}}
\setlength{\topmargin}{-1cm}
\setlength{\paperheight}{29.7cm}
\setlength{\textheight}{25.3cm}

% useful packages.
\usepackage{amsfonts}
\usepackage{amsmath}
\usepackage{amssymb}
\usepackage{amsthm}
\usepackage{enumerate}
\usepackage{graphicx}
\usepackage{multicol}
\usepackage{fancyhdr}
\usepackage{layout}
\usepackage{listings}
\usepackage{float, caption}

\lstset{
    basicstyle=\ttfamily, basewidth=0.5em
}

% some common command
\newcommand{\dif}{\mathrm{d}}
\newcommand{\avg}[1]{\left\langle #1 \right\rangle}
\newcommand{\difFrac}[2]{\frac{\dif #1}{\dif #2}}
\newcommand{\pdfFrac}[2]{\frac{\partial #1}{\partial #2}}
\newcommand{\OFL}{\mathrm{OFL}}
\newcommand{\UFL}{\mathrm{UFL}}
\newcommand{\fl}{\mathrm{fl}}
\newcommand{\op}{\odot}
\newcommand{\Eabs}{E_{\mathrm{abs}}}
\newcommand{\Erel}{E_{\mathrm{rel}}}

\begin{document}

\pagestyle{fancy}
\fancyhead{}
\lhead{李业翔, 3220103288}
\chead{数据结构与算法第四次作业}
\rhead{Oct.16th, 2024}

\section{测试程序的设计思路}

我首先创建了一个链表lst,向其中插入了元素0到9,来测试构造函数和push\_back(),顺带测试insert()。

然后,我调用了begin(),end(),++和*来打印链表,其中调用*顺带检测了retrieve() 。

之后再调用pop\_back(),pop\_front(),push\_front,push\_back,来对链表的值进行操作,其中调用pop\_back()顺带检测了--与erase。

将链表打印出来,查看效果。

测试赋值构造函数并创建lst1。

调用函数size(),empty(),front(),back()并打印结果。

将lst拷贝给lst1,检测拷贝功能。
利用拷贝构造函数创建lst2,检测拷贝构造功能。

打印lst1与lst2,查看效果。

调用clear()清空lst2,调用size()和empty()并输出结果来检验相关功能。

\section{测试的结果}

测试结果一切正常。

我用 valgrind 进行测试,发现没有发生内存泄露。

\end{document}

%%% Local Variables: 
%%% mode: latex
%%% TeX-master: t
%%% End: 
