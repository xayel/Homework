\documentclass[UTF8]{ctexart}
\usepackage{geometry, CJKutf8}
\geometry{margin=1.5cm, vmargin={0pt,1cm}}
\setlength{\topmargin}{-1cm}
\setlength{\paperheight}{29.7cm}
\setlength{\textheight}{25.3cm}

% useful packages.
\usepackage{amsfonts}
\usepackage{amsmath}
\usepackage{amssymb}
\usepackage{amsthm}
\usepackage{enumerate}
\usepackage{graphicx}
\usepackage{multicol}
\usepackage{fancyhdr}
\usepackage{layout}
\usepackage{listings}
\usepackage{float, caption}

\lstset{
    basicstyle=\ttfamily, basewidth=0.5em
}

% some common command
\newcommand{\dif}{\mathrm{d}}
\newcommand{\avg}[1]{\left\langle #1 \right\rangle}
\newcommand{\difFrac}[2]{\frac{\dif #1}{\dif #2}}
\newcommand{\pdfFrac}[2]{\frac{\partial #1}{\partial #2}}
\newcommand{\OFL}{\mathrm{OFL}}
\newcommand{\UFL}{\mathrm{UFL}}
\newcommand{\fl}{\mathrm{fl}}
\newcommand{\op}{\odot}
\newcommand{\Eabs}{E_{\mathrm{abs}}}
\newcommand{\Erel}{E_{\mathrm{rel}}}

\begin{document}

\pagestyle{fancy}
\fancyhead{}
\lhead{李业翔, 3220103288}
\chead{数据结构与算法第七次作业}
\rhead{Dec.2th, 2024}

\section{设计思路和测试流程}

1.利用make\_heap()函数和pop\_heap()函数创建heapsort()函数,实现对std::vector<typename>的heapsort排序。方法是将数组转化为堆,并不断将堆顶元素(最大值)放到堆的后面并保持剩下的元素构成堆。

2.利用<random>库来实现对长度不少于1000000的随机序列和部分元素重复序列的生成,并用for循环实现对长度不少于1000000的有序序列和逆序序列的生成。

3.通过比较相邻两个元素的大小来判断数组是否是升序的,以此实现check()函数。

4.之后就是在main()函数中对各种序列进行排序并检测排序结果是否正确。对每一种序列,需要对生成的序列进行复制,分别用heapsort()与标准库中的sortheap()进行排序并计时,以此进行效率对比。

\section{效率对比}

\begin{center}
\begin{tabular}{|c|c|c|}
	\hline
	  &  my heapsort time  &  std::sort\_heap time \\ \hline
	random sequence  &  402ms  &  387ms \\ \hline
	ordered sequence  &  416ms  &  393ms \\ \hline
	reverse sequence  &  406ms  &  394ms \\ \hline
	repetitive sequence  &  417ms  &  391ms \\ \hline
\end{tabular}
\end{center}

\section{时间复杂度分析}

在heapsort()函数中,make\_heap()被调用一次,时间复杂度为O(n)。pop\_heap()在循环中被调用n-1次(因为向量大小从 n 减小到 1),每次的时间复杂度为O(log i),其中i是当前堆的大小。将这些对数时间复杂度相加,我们得到总的时间复杂度为O((n-1)+(n-2)+...+1)=O(n log n)。heapsort()和std::sort\_heap在理论上具有相同的时间复杂度,但在实际应用中可能由于缓存局部性的影响导致std::sort\_heap效率略高。

\end{document}

%%% Local Variables: 
%%% mode: latex
%%% TeX-master: t
%%% End: 
