\documentclass[UTF8]{ctexart}
\usepackage{geometry, CJKutf8}
\geometry{margin=1.5cm, vmargin={0pt,1cm}}
\setlength{\topmargin}{-1cm}
\setlength{\paperheight}{29.7cm}
\setlength{\textheight}{25.3cm}

% useful packages.
\usepackage{amsfonts}
\usepackage{amsmath}
\usepackage{amssymb}
\usepackage{amsthm}
\usepackage{enumerate}
\usepackage{graphicx}
\usepackage{multicol}
\usepackage{fancyhdr}
\usepackage{layout}
\usepackage{listings}
\usepackage{float, caption}

\lstset{
    basicstyle=\ttfamily, basewidth=0.5em
}

% some common command
\newcommand{\dif}{\mathrm{d}}
\newcommand{\avg}[1]{\left\langle #1 \right\rangle}
\newcommand{\difFrac}[2]{\frac{\dif #1}{\dif #2}}
\newcommand{\pdfFrac}[2]{\frac{\partial #1}{\partial #2}}
\newcommand{\OFL}{\mathrm{OFL}}
\newcommand{\UFL}{\mathrm{UFL}}
\newcommand{\fl}{\mathrm{fl}}
\newcommand{\op}{\odot}
\newcommand{\Eabs}{E_{\mathrm{abs}}}
\newcommand{\Erel}{E_{\mathrm{rel}}}

\begin{document}

\pagestyle{fancy}
\fancyhead{}
\lhead{李业翔, 3220103288}
\chead{数据结构与算法第六次作业}
\rhead{Nov.11th, 2024}

\section*{remove函数的实现}

首先检查t是否为空,如果为空,则直接返回。

之后定义三个指针变量parent,current,successorParent。
其中,current指针用于遍历树以找到要删除的节点,parent指针跟踪current的父节点,successorParent指针用于处理节点有两个子节点的情况,指向找到的替代删除节点的继任节点的父节点。

用while循环寻找值为x的节点,循环结束后,若current为nullptr,说明没有找到值为x的节点,什么也不做,函数返回。找到要删除的节点后,根据节点的子节点数量来决定如何删除。

1.如果节点没有子节点,可以直接删除该节点。如果该节点是根节点(parent 为空),则将根节点设置为nullptr;否则,将父节点的相应子节点指针设置为nullptr。

2.如果节点有两个子节点,需要找到右子树中的最小节点( 继任节点) 来替换当前节点。继任节点是current->right子树中最左边的节点。找到继任节点后,先利用继任节点的父节点将继任节点删除,再将继任节点移动到当前节点的位置。

3.如果节点只有一个子节点,可以直接删除该节点,并将其子节点提升到该位置。

最后delete current,从内存中删除节点,避免内存泄漏,然后调用balance()函数,保持树的平衡。
\end{document}

%%% Local Variables: 
%%% mode: latex
%%% TeX-master: t
%%% End: 
