\documentclass[UTF8]{ctexart}
\usepackage{geometry, CJKutf8}
\geometry{margin=1.5cm, vmargin={0pt,1cm}}
\setlength{\topmargin}{-1cm}
\setlength{\paperheight}{29.7cm}
\setlength{\textheight}{25.3cm}

% useful packages.
\usepackage{amsfonts}
\usepackage{amsmath}
\usepackage{amssymb}
\usepackage{amsthm}
\usepackage{enumerate}
\usepackage{graphicx}
\usepackage{multicol}
\usepackage{fancyhdr}
\usepackage{layout}
\usepackage{listings}
\usepackage{float, caption}

\lstset{
    basicstyle=\ttfamily, basewidth=0.5em
}

% some common command
\newcommand{\dif}{\mathrm{d}}
\newcommand{\avg}[1]{\left\langle #1 \right\rangle}
\newcommand{\difFrac}[2]{\frac{\dif #1}{\dif #2}}
\newcommand{\pdfFrac}[2]{\frac{\partial #1}{\partial #2}}
\newcommand{\OFL}{\mathrm{OFL}}
\newcommand{\UFL}{\mathrm{UFL}}
\newcommand{\fl}{\mathrm{fl}}
\newcommand{\op}{\odot}
\newcommand{\Eabs}{E_{\mathrm{abs}}}
\newcommand{\Erel}{E_{\mathrm{rel}}}

\begin{document}

\pagestyle{fancy}
\fancyhead{}
\lhead{李业翔, 3220103288}
\chead{数据结构与算法第五次作业}
\rhead{Nov.4th, 2024}

\section{修改后的remove函数}

先利用findMin()创建了detachMin()函数。
在remove()函数的构建中,先用while循环找到了要删除的函数,再分两种情况讨论。
当要删除的节点有两个子节点时,先用detachMin()函数将要右子树的最小节点找出,用于代替要被删除的节点。
再通过改变节点的指针来将二叉搜索树改成预期的样子。最后删除节点,避免内存泄漏。
对于只有一个或零个子节点的情况,我保留了原来的代码。

\section{测试的结果与分析}

测试程序的思路是创建一个搜索二叉树,再调用remove()函数,来检验remove()的功能。
但最后的显示结果是均为"Empty tree",说明程序有问题,但尚不清楚原因。

\end{document}

%%% Local Variables: 
%%% mode: latex
%%% TeX-master: t
%%% End: 
