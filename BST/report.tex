\documentclass[UTF8]{ctexart}
\usepackage{geometry, CJKutf8}
\geometry{margin=1.5cm, vmargin={0pt,1cm}}
\setlength{\topmargin}{-1cm}
\setlength{\paperheight}{29.7cm}
\setlength{\textheight}{25.3cm}

% useful packages.
\usepackage{amsfonts}
\usepackage{amsmath}
\usepackage{amssymb}
\usepackage{amsthm}
\usepackage{enumerate}
\usepackage{graphicx}
\usepackage{multicol}
\usepackage{fancyhdr}
\usepackage{layout}
\usepackage{listings}
\usepackage{float, caption}

\lstset{
    basicstyle=\ttfamily, basewidth=0.5em
}

% some common command
\newcommand{\dif}{\mathrm{d}}
\newcommand{\avg}[1]{\left\langle #1 \right\rangle}
\newcommand{\difFrac}[2]{\frac{\dif #1}{\dif #2}}
\newcommand{\pdfFrac}[2]{\frac{\partial #1}{\partial #2}}
\newcommand{\OFL}{\mathrm{OFL}}
\newcommand{\UFL}{\mathrm{UFL}}
\newcommand{\fl}{\mathrm{fl}}
\newcommand{\op}{\odot}
\newcommand{\Eabs}{E_{\mathrm{abs}}}
\newcommand{\Erel}{E_{\mathrm{rel}}}

\begin{document}

\pagestyle{fancy}
\fancyhead{}
\lhead{李业翔, 3220103288}
\chead{数据结构与算法第五次作业}
\rhead{Nov.4th, 2024}

\section{修改后的remove函数}

先利用递归创建了detachMin()函数,省去了对父节点的追踪。函数的作用是返回最小节点并从原子树中删除这个节点。
remove()函数的创建是先利用递归去寻找值为x的节点,递归的好处是可以省去对父节点的追踪。
对于找不到值为x的节点的情况,即条件t==nullptr触发,remove()函数不做任何操作。
若找到了值为x的节点,分两种情况讨论删除节点的方法。
当节点有两个子树时,我们需要用右子树的最小节点去替换要删除的节点,方式是通过节点指针的修改。值得注意的是,由于t是父节点左右指针之一的引用,所以我们直接修改t就行,而不用考虑父节点的位置。
当节点只有一个子树时,父节点指针直接指向要删除的节点的下一个节点即可。
最后,在两种情况中,都需要在将节点从树中删除后,用delete将节点从内存中删除,从而避免内存泄漏。

\section{测试的结果与分析}

测试结果一切正常,也没有发生内存泄漏。

\end{document}

%%% Local Variables: 
%%% mode: latex
%%% TeX-master: t
%%% End: 
